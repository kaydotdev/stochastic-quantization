%% The first command in your LaTeX source must be the \documentclass command.

\documentclass{ceurart}

\usepackage{algorithm}
\usepackage{algpseudocode}
\usepackage{amsthm}
\usepackage{todonotes}

%%
%% One can fix some overfulls
\sloppy

%%
%% Minted listings support 
\usepackage{listings}
%% auto break lines
\lstset{breaklines=true}

%%
%% end of the preamble, start of the body of the document source.
\begin{document}

%%
%% The "title" command
\title{}

%%
%% The "author" command and its associated commands are used to define
%% the authors and their affiliations.
\author[1,2]{Vladimir Norkin}[%
    orcid=0000-0003-3255-0405
]
\author[2]{Anton Kozyriev}[%
    orcid=0009-0007-6692-2162,
    email=a.kozyriev@kpi.ua
]

\address[1]{V.M. Glushkov Institute of Cybernetics of the NAS of Ukraine, Kyiv}
\address[2]{National Technical University of Ukraine "Igor Sikorsky Kyiv Polytechnic Institute", Kyiv}

%%
%% The abstract is a short summary of the work to be presented in the
%% article.
\begin{abstract}
  Abstract
\end{abstract}

%%
%% Keywords. The author(s) should pick words that accurately describe
%% the work being presented. Separate the keywords with commas.
\begin{keywords}
  keyword 1 \sep
  keyword 2 \sep
  keyword 3 \sep
  keyword 4
\end{keywords}

%%
%% This command processes the author and affiliation and title
%% information and builds the first part of the formatted document.
\maketitle

\section{Introduction}

\section{Two stage classification}

\section{Deep metric learning}

\subsection{Convolutional neural network}

\subsection{Backpropagation}

\subsection{Triplet loss}

\section{Stochastic quantization}

\subsection{Non-convex problems of stochastic optimization}

\subsection{Transport problem based on KR-distance}

\subsection{Comparison with KMeans}

\subsection{Proposed stochastic quantization}

\subsection{Modifications of the algorithm}

\section{Numerical experiments}

\section{Conclusions}

%%
%% Define the bibliography file to be used
\bibliography{references}

\end{document}

%%
%% End of file
